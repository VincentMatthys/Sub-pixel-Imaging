\documentclass[12pt,a4paper,onecolumn]{article}
\input{packages}
\setcounter{section}{5} % to start counting section to 6


% ------------------------ General informations --------------------------------
\title{Sub_pixel_image_processing_tp_6}
\author{Vincent Matthys}
\graphicspath{{images/}}
% ------------------------------------------------------------------------------


\begin{document}

\begin{tabularx}{0.9\textwidth}{@{} l X r @{} }
	{\textsc{Master MVA}}               &  & \textsc{TP6}       \\
	\textsc{Sub-pixel image processing} &  & {ENS Paris Saclay} \\
\end{tabularx}
\vspace{1.5cm}
\begin{center}

	\rule[11pt]{5cm}{0.5pt}

	\textbf{\LARGE \textsc{Compte-rendu TP6}}
	\vspace{0.5cm}

	Vincent Matthys

	vincent.matthys@ens-paris-saclay.fr

	\rule{5cm}{0.5pt}

	\vspace{1.5cm}
\end{center}


\setcounter{section}{14}
\section{Exerice 15}


\setcounter{section}{17}
\section{Exerice 18}

\subsection{}

On peut réecrire le noyau d'interpolation \(\phi\) suivant :

\begin{equation}
	\begin{split}
		\phi(x) &= sinc(x) \times \mathds{1}_{[-p, p]}(x)\\
		\hat{\phi}(\xi) &= \mathcal{F}\left(sinc(x) \times \mathds{1}_{[-p, p]}(x)\right)\\
		&= \hat{U}(2\xi)\\
		&=  \int_{\mathbb{R}}\sum_{k\in\mathbb{Z}} u[k]sinc(x-k) e^{-ix(2\xi)} \quad \text{d'après \eqref{th_shannon}}\\
		&= \sum_{k\in\mathbb{Z}} u[k] e^{-ik(2\xi)}\int_{\mathbb{R}} sinc(x-k) e^{-i(x-k)(2\xi)}dx\\
		&= \sum_{k\in\mathbb{Z}} u[k] e^{-ik(2\xi)} \mathds{1}_{[-\pi, \pi]}(2\xi)\\
		&= \hat{u}(2\xi)\mathds{1}_{[-\pi, \pi]}(2\xi)\\
	\end{split}
\end{equation}


\end{document}
