\documentclass[12pt,a4paper,onecolumn]{article}
\input{packages}

% ------------------------ General informations --------------------------------
\title{Matrice Fondamentale}
\author{Vincent Matthys}
\graphicspath{{images/}}
% ------------------------------------------------------------------------------

% \renewcommand{\thesubsection}{\alph{subsection}}


\begin{document}

\begin{tabularx}{0.9\textwidth}{@{} l X r @{} }
	{\textsc{Master MVA}}               &  & \textsc{TP2}       \\
	\textsc{Sub-pixel image processing} &  & {ENS Paris Saclay} \\
\end{tabularx}
\vspace{1.5cm}
\begin{center}

	\rule[11pt]{5cm}{0.5pt}

	\textbf{\LARGE \textsc{Compte-rendu TP3}}
	\vspace{0.5cm}

	Vincent Matthys

	vincent.matthys@ens-paris-saclay.fr

	\rule{5cm}{0.5pt}

	\vspace{1.5cm}
\end{center}

\section{Objectifs}
Le programme permet à l'utilisateur :
\begin{enumerate}
	\item de déterminer la matrice fondamentale à partir de deux images d'une même scène sans connaissance au préalable des paramètres internes de l'appareil
	\item déterminer les \textit{SIFT inliers}
	\item de tracer la ligne épipolaire associé à un point sélectionné par l'utilisateur dans l'autre image
\end{enumerate}


\end{document}
