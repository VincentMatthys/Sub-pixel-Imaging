\documentclass[12pt,a4paper,onecolumn]{article}
\input{packages}
\setcounter{section}{5} % to start counting section to 6


% ------------------------ General informations --------------------------------
\title{Matrice Fondamentale}
\author{Vincent Matthys}
\graphicspath{{images/}}
% ------------------------------------------------------------------------------


\begin{document}

\begin{tabularx}{0.9\textwidth}{@{} l X r @{} }
	{\textsc{Master MVA}}               &  & \textsc{TP2}       \\
	\textsc{Sub-pixel image processing} &  & {ENS Paris Saclay} \\
\end{tabularx}
\vspace{1.5cm}
\begin{center}

	\rule[11pt]{5cm}{0.5pt}

	\textbf{\LARGE \textsc{Compte-rendu TP3}}
	\vspace{0.5cm}

	Vincent Matthys

	vincent.matthys@ens-paris-saclay.fr

	\rule{5cm}{0.5pt}

	\vspace{1.5cm}
\end{center}

\section{Exercice 6}

\begin{figure}[H]
	\centering
	\begin{subfigure}[b]{\textwidth}
		\centering
		\includegraphics[height = 0.25\textheight]{6_5}
		\subcaption{Pour \(N = 5\)}
		\label{fig_6_5}
	\end{subfigure}
	\begin{subfigure}[b]{\textwidth}
		\centering
		\includegraphics[height = 0.25\textheight]{6_10}
		\subcaption{Pour \(N = 10\)}
		\label{fig_6_10}
	\end{subfigure}
	\begin{subfigure}[b]{\textwidth}
		\centering
		\includegraphics[height = 0.25\textheight]{6_50}
		\subcaption{Pour \(N = 50\)}
		\label{fig_6_50}
	\end{subfigure}
	\caption{Représentation graphique de la convergence du \(sincd_N\) et du \(sinc\), sur \([-N/2, N/2]\)}
	\label{fig_6}
\end{figure}

En figure~\ref{fig_6} sont représentés le \(sinc\) et le \(sincd_N\), pour différentes valeurs de \(N\) pour \(x \in[-N/2, N/2]\). On constate que, pour chaque valeur entière, on a bien égalité, et que, pour N assez grand, les différences sont minimes.

\section{Exercice 7}


\end{document}
