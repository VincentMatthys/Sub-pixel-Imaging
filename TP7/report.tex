\documentclass[12pt,a4paper,onecolumn]{article}
\usepackage[utf8]{inputenc}
\usepackage[T1]{fontenc}
\usepackage[french]{babel}

% ------------------------- Color table ----------------------------------------
\usepackage{multirow}
\usepackage[table]{xcolor}
\definecolor{maroon}{cmyk}{0,0.87,0.68,0.32}
% ------------------------------------------------------------------------------

\usepackage{amscd}
\usepackage{amsthm}
\usepackage{physics}
\usepackage[left=2.2cm,right=2.2cm,top=2cm,bottom=2cm]{geometry}
\usepackage{textcomp,gensymb} %pour le °C, et textcomp pour éviter les warning
\usepackage{graphicx} %pour les images
\usepackage{caption}
\usepackage{subcaption}
\usepackage[colorlinks=true,
	breaklinks=true,
	citecolor=blue,
	linkcolor=blue,
	urlcolor=blue]{hyperref} % pour insérer des liens
\usepackage{epstopdf} %converting to PDF
\usepackage[export]{adjustbox} %for large figures

\usepackage{array}
\usepackage{dsfont}% indicatrice : \mathds{1}


% -------------------------- Mathematics ---------------------------------------
\graphicspath{{images/}} % For the images path
% ------------------------------------------------------------------------------

% -------------------------- Mathematics ---------------------------------------
\usepackage{mathrsfs, amsmath, amsfonts, amssymb}
\usepackage{bm}
\usepackage{mathtools}
\usepackage[Symbol]{upgreek} % For pi \uppi different from /pi
\newcommand{\R}{\mathbb{R}} % For Real space
% ------------------------------------------------------------------------------


% -------------------------- Code format ---------------------------------------
\usepackage[numbered,framed]{matlab-prettifier}
\lstset{
	style              = Matlab-editor,
	basicstyle         = \mlttfamily,
	escapechar         = '',
	mlshowsectionrules = true,
}
% ------------------------------------------------------------------------------

% ------------------------- Blbiographie --------------------------------------
% \usepackage[backend=biber, style=science]{biblatex}
% \addbibresource{biblio.bib}
% ------------------------------------------------------------------------------


\setcounter{tocdepth}{4} %Count paragraph
\setcounter{secnumdepth}{4} %Count paragraph
\usepackage{float}

\usepackage{graphicx} % for graphicspath
% \graphicspath{{../images/}}

\usepackage{array,tabularx}
\newcolumntype{L}[1]{>{\raggedright\let\newline\\\arraybackslash\hspace{0pt}}m{#1}}
\newcolumntype{C}[1]{>{\centering\let\newline\\\arraybackslash\hspace{0pt}}m{#1}}
\newcolumntype{R}[1]{>{\raggedleft\let\newline\\\arraybackslash\hspace{0pt}}m{#1}}

\newcommand{\assignmenttitle}{}
\newcommand{\studentname}{}
\newcommand{\email}{}
\newcommand{\schoolyear}{2017/2018}


\title{
\normalfont \normalsize 
\textsc{Object recognition and computer vision, Master MVA, \schoolyear} \\
[10pt] 
\rule{\linewidth}{0.5pt} \\[6pt] 
\huge \assignmenttitle \\
\rule{\linewidth}{2pt}  \\[10pt]
}

\author{\studentname}

\date{\small\email}

\newcommand{\question}[1]{\subsubsection*{#1}}

\setlist[enumerate]{topsep=0pt,itemsep=-1ex,partopsep=1ex,parsep=1ex,label=(\roman*)}

\graphicspath{{images/}}

\newcommand{\labelnotempty}[1]{
\def\temp{#1}\ifx\temp\empty
\else
    \label{#1}
\fi
}
% single figure
\newcommand{\singlefig}[4]{
\begin{figure}[ht!]
        \centering
        \includegraphics[width={#2}\columnwidth]{#1}
        \caption{#3}
        \labelnotempty{#4}
\end{figure}}

\newcommand{\subfig}[4]{
\includegraphics[width={#2}\columnwidth]{#1}
\caption{#3}
\labelnotempty{#4}
}

% double figure
\newcommand{\doublefig}[4]{
\begin{figure}[ht!]
    \centering
    \begin{subfigure}[t]{0.45\columnwidth}
        \centering
    #1
    \end{subfigure}
    ~
    \begin{subfigure}[t]{0.45\columnwidth}
        \centering
    #2
    \end{subfigure}
    \caption{#3}
    \labelnotempty{#4}
\end{figure}}

% triple figure
\newcommand{\triplefig}[5]{
\begin{figure}[ht!]
    \centering
    \begin{subfigure}[t]{0.30\columnwidth}
        \centering
    #1
    \end{subfigure}
    ~
    \begin{subfigure}[t]{0.30\columnwidth}
        \centering
    #2
    \end{subfigure}
    ~
    \begin{subfigure}[t]{0.30\columnwidth}
        \centering
    #3
    \end{subfigure}
    \caption{#4}
    \labelnotempty{#5}
\end{figure}}



% ------------------------ General informations --------------------------------
\title{MVA - Sub-pixel image processing - TP7}
\author{Vincent Matthys}
\graphicspath{{images/}{../images/}} % For the images path
% ------------------------------------------------------------------------------


\begin{document}

\begin{tabularx}{0.9\textwidth}{@{} l X r @{} }
	{\textsc{Master MVA}}               &  & \textsc{Vincent Matthys} \\
	\textsc{Sub-pixel image processing} &  & {ENS Paris Saclay}       \\
\end{tabularx}
\vspace{1.5cm}
\begin{center}

	\rule[11pt]{5cm}{0.5pt}

	\textbf{\LARGE \textsc{Compte rendu - TP7}}
	\vspace{0.5cm}

	Vincent Matthys

	vincent.matthys@ens-paris-saclay.fr

	\rule{5cm}{0.5pt}

	\vspace{1.5cm}
\end{center}


\setcounter{section}{19}
\section{Exercice 20}

\subsection{}

Du fait de l'interpolation exacte, on doit avoir :

\begin{equation*}
	\begin{aligned}
		u[l] = v[l] & = \sum_{k\in\mathbb{Z}}c[k]\psi[l-k] \quad \forall l \in \mathbb{Z} \\
		u           & = c * \psi_d                                                        \\
	\end{aligned}
\end{equation*}

où \(\psi_d\) est la version discrétisée de \(\psi\). Ceci permet d'établir la relation entre les coefficients de Fourier \(\widehat{u}\), \(\widehat{c}\) et \(\widehat{\psi_d}\) suivante :

\begin{equation*}
	\widehat{u}(\xi) = \widehat{c}(\xi) \times \widehat{\psi_d}(\xi) \quad \text{avec} \quad \widehat{\psi_d}(\xi) = \sum_{n \in \mathbb{Z}}\widehat{\psi}(\xi + 2n\pi)
\end{equation*}
puisque discrétiser avec un pas de 1 \(\psi\) revient à \(2\pi\)~périodiser sa transformée de Fourier.

Il faut alors remarquer que \(\psi\) peut s'écrire sous la forme suivante :
\begin{equation*}
	\begin{aligned}
		\psi(x) = \left(\mathds{1}_{[\alpha - 1/2, \alpha - 1/2]} * \mathds{1}_{[\alpha - 1/2, \alpha - 1/2]}\right)(x) \quad \forall x \in \mathbb{R}
	\end{aligned}
\end{equation*}

ce qui permet d'expiliciter rapidement la transformée de Fourier de \(\psi\) :

\begin{equation*}
	\begin{aligned}
		\widehat{\psi}(\xi) & = \widehat{\mathds{1}_{[\alpha - 1/2, \alpha - 1/2]}}(\xi) \times \widehat{\mathds{1}_{[\alpha - 1/2, \alpha - 1/2]}}(\xi) \quad \forall \xi \in \mathbb{R} \\
		\widehat{\psi}(\xi) & = \left(e^{-i\alpha\xi}~sinc\left(\frac{\xi}{2}\right)\right)^2           \forall \xi \in \mathbb{R}^* \quad \text{et} \quad \widehat{\psi}(0) = 1          \\
		\widehat{\psi}(\xi) & = e^{-2i\alpha\xi}\left(sinc\left(\frac{\xi}{2}\right)\right)^2                                                                                             \\
	\end{aligned}
\end{equation*}

On peut alors en déduire l'expression de la transformée de Fourier de \(\psi\) discrétisée :

\begin{equation*}
	\begin{aligned}
		\widehat{\psi_d}(\xi) & = \sum_{n \in \mathbb{Z}}\widehat{\psi}(\xi + 2n\pi)                                                     \\
		\widehat{\psi_d}(\xi) & = \sum_{n \in \mathbb{Z}}e^{-2i\alpha(\xi + 2n\pi)}\left(sinc\left(\frac{\xi + 2n\pi}{2}\right)\right)^2 \\
		\widehat{\psi_d}(\xi) & = e^{-2i\alpha\xi}\sum_{n \in \mathbb{Z}}\left(\frac{\sin(\xi / 2 + n\pi)}{\xi / 2 + n\pi}\right)^2      \\
		\widehat{\psi_d}(\xi) & = e^{-2i\alpha\xi}\sum_{n \in \mathbb{Z}}\left(\frac{\sin(\xi / 2)(-1)^n}{\xi / 2 + n\pi}\right)^2       \\
		\widehat{\psi_d}(\xi) & = e^{-2i\alpha\xi}\sin^2(\xi / 2)\sum_{n \in \mathbb{Z}}\left(\frac{1}{\xi / 2 + n\pi}\right)^2          \\
	\end{aligned}
\end{equation*}

Et si on considère alors \(\widehat{\psi_d}\) sur une période \(2\pi\), à savoir sur \([-\pi, \pi]\) :
\begin{equation*}
	\begin{aligned}
		\lvert\widehat{\psi_d}(\xi)\rvert & = \sin^2(\xi / 2)\sum_{n \in \mathbb{Z}}\left(\frac{1}{\xi / 2 + n\pi}\right)^2 > 0 \quad \forall \xi \in [-\pi, \pi] \setminus \{0\} \\
		\lvert \widehat{\psi_d}(0) \rvert & = 1 > 0                                                                                                                               \\
	\end{aligned}
\end{equation*}
on prouve que \(\widehat{\psi_d}\) ne s'annule jamais, ce qui permet finalement d'exprimer \(\widehat{c}\) en fonction de \(\widehat{u}\) :
\begin{equation*}
	\begin{aligned}
		\widehat{c}(\xi) & = \frac{\widehat{u}(\xi)}{\widehat{\psi_d}(\xi)} \quad \forall \xi \in \mathbb{R} \\
	\end{aligned}
\end{equation*}
\begin{equation*}
	\widehat{c}(\xi) =\left\{
	\begin{aligned}
		 & \widehat{u}(0) \quad \text{pour } \quad \xi = 0                                                                                                          \\
		 & \frac{\widehat{u}(\xi)e^{2i\alpha\xi}}{\sin^2(\xi / 2)\sum_{n \in \mathbb{Z}}\left(\frac{1}{\xi / 2 + n\pi}\right)^2} \quad \forall \xi \in \mathbb{R^*} \\
	\end{aligned}
	\right.
\end{equation*}

\subsection{}


\end{document}
